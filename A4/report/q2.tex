\section{Forecasting on a Real World Dataset}
\subsection{2.1}
\subsubsection{Part (a)}
Our data has
\begin{itemize}[nosep]
	\item Trend
	\item Seasonality (3 months)
	\item Varying variance
\end{itemize}

To remove varying variance I took log of the data, then took difference with
lag 3 to remove seasonality.
\subsubsection{Training the model}
Since there's a clear seasonality, I added a \texttt{SARIMA} (seasonal \texttt{ARIMA}) model.
I'm using the parameters which can by easily deduced using \texttt{ACF} and \texttt{PACF} plots
of the preprocessed data. Also we evaluate errors (\texttt{MASE} and \texttt{MAPE}).
\subsubsection{Using and Interpreting our Model}
Finally, we solve the question! Predicting the number of \texttt{PASSENGERS CARRIED}
from 2023 September to 2024 August. Since we added some preprocessing to our
data, we need to reverse it. Here are the things we did
\begin{itemize}[nosep]
	\item Took log of the data
	\item Took difference with lag 3 (season)
\end{itemize}

So I first reverse the difference by adding data at 3 values before. Then I
reverse the log by taking exp of the data.

\begin{figure}[H]
	\centering
	\includegraphics[width=0.8\textwidth]{images/2a.png}
\end{figure}

\subsubsection{Part (b)}

This is the prompt given:
\begin{center}
	\texttt{
		Given the following monthly airline passenger data, predict and display the
		values for next 12 months (2023 SEP to 2024 AUG) for Passengers Carried, you
		need not show any code or your though process, just the final predicted values
		must be displayed, there's a season of 4 months: Airline A007, Year 2023,
		Month JAN, Passengers Carried: 6847384.0.
	}
\end{center}

The final evaluations by the LLM GPT4o are:

\begin{center}
	\begin{tabular}{|c|c|}
		\hline
		Year \& Month   & Passengers Carried \\
		\hline
		\texttt{23 SEP} & \texttt{7789024}   \\
		\texttt{23 OCT} & \texttt{8122317}   \\
		\texttt{23 NOV} & \texttt{8225536}   \\
		\texttt{23 DEC} & \texttt{8504783}   \\
		\texttt{24 JAN} & \texttt{7739815}   \\
		\texttt{24 FEB} & \texttt{7631294}   \\
		\texttt{24 MAR} & \texttt{8124572}   \\
		\texttt{24 APR} & \texttt{8218501}   \\
		\texttt{24 MAY} & \texttt{8802926}   \\
		\texttt{24 JUN} & \texttt{8550792}   \\
		\texttt{24 JUL} & \texttt{8289312}   \\
		\texttt{24 AUG} & \texttt{8494387}   \\
		\hline
	\end{tabular}
\end{center}

\subsubsection{Part (c)}

After applying the prophet model, the final predictions are:

\begin{center}
	\begin{tabular}{|c|c|}
		\hline
		Year \& Month   & Passengers Carried    \\
		\hline
		\texttt{23 SEP} & \texttt{1.316906e+06} \\
		\texttt{23 OCT} & \texttt{2.067208e+06} \\
		\texttt{23 NOV} & \texttt{2.391833e+06} \\
		\texttt{23 DEC} & \texttt{2.002217e+06} \\
		\texttt{24 JAN} & \texttt{2.112962e+06} \\
		\texttt{24 FEB} & \texttt{2.170063e+06} \\
		\texttt{24 MAR} & \texttt{2.191516e+06} \\
		\texttt{24 APR} & \texttt{2.404104e+06} \\
		\texttt{24 MAY} & \texttt{2.605715e+06} \\
		\texttt{24 JUN} & \texttt{2.192019e+06} \\
		\texttt{24 JUL} & \texttt{2.120529e+06} \\
		\texttt{24 AUG} & \texttt{2.404104e+06} \\
		\hline
	\end{tabular}
\end{center}


\subsection{2.2}
In forecasting demand for fleet management and human resource planning, Mean
Absolute Percentage Error (\texttt{MAPE}) may not be the best metric for evaluation.
This is because
\begin{itemize}
	\item \textbf{Sensitivity to Low Passenger Volumes:}  \texttt{MAPE} can inflate
	      errors in months or seasons with low passenger volumes because it
	      calculates percentage error. For fleet management, which focuses on
	      total passenger volume over a quarter, these small-volume periods
	      can skew \texttt{MAPE} disproportionately and lead to misleading conclusions
	      about capacity requirements.
	\item \textbf{Human Resource Needs Based on Peak Demand:} For staffing,
	      peak demand periods are often more critical than average levels.
	      \texttt{MAPE} does not account for these peak demands, which are essential
	      to ensuring adequate staffing during high-demand times.
\end{itemize}

The metric that could be better for this case is \textbf{Mean Absolute Scaled
	Error (\texttt{MASE})} due to the following reasons:
\begin{itemize}
	\item \texttt{MASE} is scale-invariant, making it useful across high- and
	      low-demand periods without the issues \texttt{MAPE} has in low-volume
	      situations.
	\item By scaling against average demand from a baseline period, \texttt{MASE}
	      provides a clearer picture of prediction error across fluctuating
	      demands, making it helpful for capturing both total and peak
	      periods—key for both fleet and human resources needs.
\end{itemize}

\subsection{2.3}
Given that $\Delta Y= $ (first-differenced series) is weakly stationary and can
be represented as:
\begin{equation}
	\Delta Y = \mu + \mathcal{N}(0,\sigma)
\end{equation}
where $\sigma$ is known and $\mu$ is an unknown constant, we are tasked with
testing if $\mu$ differs between the pre-COVID (before December 2019) and
post-COVID (after January 2022) periods.
\\
We can use a \textbf{Two-Sample t-test} for comparing means of $\mu$ across the
two periods. Given $\sigma$ is known and assuming normal distribution of
$\Delta Y$, the two-sample t-test will allow us to test if the mean demand
(represented by $\mu$) significantly changed between the pre- and post-COVID
periods.
