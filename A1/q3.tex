% Start writing your answer from here, if you want to use new packages/change something do it in main.tex
\begin{que}
		\textbf{3.1} Let $Q_{1}$, $Q_{2}$ be non-negative random variables. Let $P(Q_1 < q_1) \geq 1-p_1$ and $P(Q_2 < q_2) \geq 1-p_2$
		where $q_1, q_2$ are non-negative. Then show that $P(Q_1Q_2 < q_1q_2) \geq 1 - (p_1 + p_2)$\\
		\textbf{3.2} Given n distinct values ${\{x_i\}}^n_{i=1}$ with mean $\mu$ and standard deviation $\sigma$, prove that for all $i$,
		we have $|x_i - \mu| \leq \sigma \sqrt[]{n-1}$. How does this inequality compare with Chebyshev's inequality as n
		increases? (give an informal answer)
	

	\hspace*{\fill} [5 marks]
\end{que}

\begin{tcolorbox}[breakable]
	\begin{sol}
		\textbf{3.1}Define two events, $E_1$ and $E_2$:
		\begin{enumerate}
			\item $E_1=\{Q_1<q_1\}$
			\item $E_2=\{Q_2<q_2\}$
		\end{enumerate}
		So,
		\[P(E_1)\geq 1-p_1\]
		\[P(E_2)\geq 1-p_2\]
		We need to prove that,
		\[ P(Q_1Q_2 < q_1q_2) \geq 1 - (p_1 + p_2)\]
		Let's define another event $E_3$, where $E_3=\{Q_1Q_2 < q_1q_2\}$\\
		If we consider the complement of $E_3$, which is
		\[\{Q_1Q_2 < q_1q_2\}^\complement=\{Q_1Q_2 \geq q_1q_2\}\]
		\[E_3^\complement=\{Q_1Q_2 \geq q_1q_2\}\]
		\clearpage
		If $Q_1Q_2 \geq q_1q_2$, and $Q_1,Q_2$ are non-negative integers, it is very clear that, atleast one of the following has to be true:
		\[Q_1\geq q_1 \text{ or }Q_2\geq q_2 \]
		This means \[E_3^\complement \subseteq \{Q_1\geq q_1\}\cup\{Q_2\geq q_2\} \]
		This implies,
		\[P(E_3^\complement)\leq P(\{Q_1\geq q_1\}\cup\{Q_2\geq q_2\} )\]
		\[P(E_3^\complement)\leq P(\{Q_1\geq q_1\})+P(\{Q_2\geq q_2\} )\]
		It is clear that:
		\[\{Q_1\geq q_1\} = E_1^\complement \text{ and } \{Q_2\geq q_2\} = E_2^\complement\]
		So,
		\[1-P(E_3)\leq P(E_1^\complement) + P(E_2^\complement)\]
		\[1-P(E_3)\leq 1-P(E_1) + 1-P(E_2)\]
		\[1-P(E_3)\leq 1-(1-p_1) + 1-(1-p_2)\]
		\[1-P(E_3)\leq p_1 +p_2\]
		This implies,
		\[P(E_3)\geq 1-(p_1+p_2)\]
		\textbf{3.2} We know that,
		\[\frac{\sum^n_{i=0}(x_i-\mu)^2}{n-1}=\sigma^2\]
		This implies,
		\[\sum^n_{i=0}(x_i-\mu)^2 = \sigma^2(n-1)\]
		For any $i$,
		\[(x_i-\mu)^2\geq0\]
		So, for each $i$,
		\[(x_i-\mu)^2\leq\sigma^2\times(n-1)\]
		Again, as both $(x_i-\mu)^2$ and $\sigma^2\times(n-1)$ are greater than or equal to zero, we can take square root on both sides.
		\[\sqrt[2]{(x_i-\mu)^2}\leq\sqrt[2]{\sigma^2(n-1)}\]
		\[|x_i-\mu|\leq\sigma\sqrt[]{n-1}\]


	\end{sol}
\end{tcolorbox}
