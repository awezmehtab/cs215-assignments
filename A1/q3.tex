% Start writing your answer from here, if you want to use new packages/change something do it in main.tex
\begin{que}
	This question has two parts:
	\begin{itemize}
		\item \textbf{3.1} Let $Q_{1}$, $Q_{2}$ be non-negative random variables. Let $P(Q_1 < q_1) \geq 1-p_1$ and $P(Q_2 < q_2) \geq 1-p_2$
		      where $q_1, q_2$ are non-negative. Then show that $P(Q_1Q_2 < q_1q_2) \geq 1 - (p_1 + p_2)$\\
		      \hspace*{\fill} [3 marks]
		\item \textbf{3.2} Given n distinct values ${\{x_i\}}^n_{i=1}$ with mean $\mu$ and standard deviation $\sigma$, prove that for all $i$,
		      we have $|x_i - \mu| \leq \sigma \sqrt[]{n-1}$. How does this inequality compare with Chebyshev's inequality as n
		      increases? (give an informal answer)
	\end{itemize}
	\hspace*{\fill} [3+2 marks]
\end{que}

\begin{tcolorbox}[breakable]
	\begin{sol}

		\textbf{3.1}

		Define two events, $E_1$ and $E_2$:
		\begin{enumerate}
			\item $E_1=\{Q_1<q_1\}$
			\item $E_2=\{Q_2<q_2\}$
		\end{enumerate}
		So,
		\begin{align*}
			P(E_1)\geq 1-p_1 \\
			P(E_2)\geq 1-p_2
		\end{align*}
		We need to prove that,
		\begin{align*}
			P(Q_1Q_2 < q_1q_2) \geq 1 - (p_1 + p_2)
		\end{align*}
		Let's define another event $E_3$, where $E_3=\{Q_1Q_2 < q_1q_2\}$\\
		If we consider the complement of $E_3$, which is
		\begin{align*}
			\{Q_1Q_2 < q_1q_2\}^\complement=\{Q_1Q_2 \geq q_1q_2\} \\
			E_3^\complement=\{Q_1Q_2 \geq q_1q_2\}
		\end{align*}
		If $Q_1Q_2 \geq q_1q_2$, and $Q_1,Q_2$ are non-negative integers, it is very clear that, atleast one of the following has to be true:
		\begin{equation*}
			Q_1\geq q_1 \text{ or }Q_2\geq q_2 \\
		\end{equation*}
		\begin{align*}
			 & \implies E_3^\complement \subseteq \{Q_1\geq q_1\}\cup\{Q_2\geq q_2\}
		\end{align*}
		\begin{align*}
			\implies P(E_3^\complement) & \leq P\left(\{Q_1\geq q_1\}\cup\{Q_2\geq q_2\} \right) \\
			                            & \leq P(\{Q_1\geq q_1\})+P(\{Q_2\geq q_2\} )
		\end{align*}
		It is clear that:
		\begin{align*}
			\{Q_1\geq q_1\} = E_1^\complement \text{ and } \{Q_2\geq q_2\} = E_2^\complement
		\end{align*}
		So,
		\begin{align*}
			1-P(E_3)        & \leq P(E_1^\complement) + P(E_2^\complement) \\
			                & \leq 1-P(E_1) + 1-P(E_2)                     \\
			                & \leq 1-(1-p_1) + 1-(1-p_2)                   \\
			                & \leq p_1 +p_2                                \\
			\implies P(E_3) & \geq 1-(p_1+p_2)
		\end{align*}


		\textbf{Solution 3.2}
		We know that,
		\begin{align*}
			\frac{\sum^n_{i=0}(x_i-\mu)^2}{n-1}=\sigma^2 \\
			\implies \sum^n_{i=0}(x_i-\mu)^2 = \sigma^2(n-1)
		\end{align*}
		For any $i$,
		\begin{align*}
			(x_i-\mu)^2\geq0
		\end{align*}
		So, for each $i$,
		\begin{align*}
			(x_i-\mu)^2\leq\sigma^2\times(n-1)
		\end{align*}
		Again, as both $(x_i-\mu)^2$ and $\sigma^2\times(n-1)$ are greater than or equal to zero, we can take square root on both sides.
		\begin{align*}
			\sqrt[2]{(x_i-\mu)^2}\leq\sqrt[2]{\sigma^2(n-1)}
			|x_i-\mu|\leq\sigma\sqrt[]{n-1}
		\end{align*}
		Chebyshev's inequality states that for any distribution, the proportion of values that lie more than $k$ standard deviations away from the mean is bounded by:
		\[P(|x-\mu | \geq k\sigma\leq \frac{1}{k^2})\]
		As $n$ increases, the deterministic bound $|x_i-\mu|\leq \sigma\sqrt[]{n-1}$ can potentially become less tight as $\sqrt[]{n-1}$, but it remains a stronger bound for small datasets compared to Chebyshev's inequality. As Chebyshev's inequality, on the other hand, remains a probabilistic statement that applies to a portion of the distribution and becomes less restrictive for larger values of $k$.
	\end{sol}
\end{tcolorbox}
