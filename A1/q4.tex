% Start writing your answer from here, if you want to use new packages/change something do it in main.tex
\begin{que}
	You need a new staff assistant, and you have n people to interview. You
	want to hire the best candidate for the position. When you interview a
	candidate, you can give them a score, with the highest score being the
	best and no ties being possible.

	You interview the candidates one by one. Because of your company’s
	hiring practices, after you interview the kth candidate, you either
	offer the candidate the job before the next interview or you forever
	lose the chance to hire that candidate. We suppose the candidates are
	interviewed in a random order, chosen uniformly at random from all n!
	possible orderings.

	We consider the following strategy. First, interview m candidates but
	reject them all: these candidates give you an idea of how strong the
	field is. After the mth candidate. hire the first candidate you
	interview who is better than all of the previous candidates you have
	interviewed.

	\begin{enumerate}
		\item Let $E$ be the event that we hire the best assistant, and let $E_i$;
		      be the event that ith candidate is the best and we hire him.
		      Determine $Pr(E_i)$, and show that
		      \begin{align}
			      Pr(E) = \frac{m}{n}\sum_{j=m+1}^{n} \frac{1}{j-1}
		      \end{align}
		      \hspace*{\fill}[4 marks]
		\item Bound $\sum_{j=m+1}^n \frac{1}{j-1}$ to obtain:
		      \begin{align}
			      \frac{m}{n}(\ln{n} - \ln{m}) \leq Pr(E) \leq
			      \frac{m}{n}(\ln(n-1) - \ln(m-1))
		      \end{align}
		      \hspace*{\fill}[3 marks]
		\item Show that $\frac{m}{n}(\ln(n) - \ln(m))$ is maximized
		      when $m = \frac{n}{e}$ , and explain why this means $Pr(E)
			      \geq \frac{1}{e}$ for this choice of $m$.
		      \hspace*{\fill}[3 marks]
	\end{enumerate}
\end{que}

\begin{tcolorbox}
	\begin{sol}

	\end{sol}
\end{tcolorbox}
