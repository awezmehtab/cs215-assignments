% Start writing your answer from here, if you want to use new packages/change something do it in main.tex
\begin{que}
	You need a new staff assistant, and you have n people to interview. You
	want to hire the best candidate for the position. When you interview a
	candidate, you can give them a score, with the highest score being the
	best and no ties being possible.

	You interview the candidates one by one. Because of your company’s
	hiring practices, after you interview the kth candidate, you either
	offer the candidate the job before the next interview or you forever
	lose the chance to hire that candidate. We suppose the candidates are
	interviewed in a random order, chosen uniformly at random from all n!
	possible orderings.

	We consider the following strategy. First, interview m candidates but
	reject them all: these candidates give you an idea of how strong the
	field is. After the mth candidate. hire the first candidate you
	interview who is better than all of the previous candidates you have
	interviewed.

	\begin{enumerate}
		\item Let $E$ be the event that we hire the best assistant, and let $E_i$;
		      be the event that ith candidate is the best and we hire him.
		      Determine $Pr(E_i)$, and show that
		      \begin{align}
			      Pr(E) = \frac{m}{n}\sum_{j=m+1}^{n} \frac{1}{j-1}
		      \end{align}
		      \hspace*{\fill}[4 marks]
		\item Bound $\sum_{j=m+1}^n \frac{1}{j-1}$ to obtain:
		      \begin{align}
			      \frac{m}{n}(\ln{n} - \ln{m}) \leq Pr(E) \leq
			      \frac{m}{n}(\ln(n-1) - \ln(m-1))
		      \end{align}
		      \hspace*{\fill}[3 marks]
		\item Show that $\frac{m}{n}(\ln(n) - \ln(m))$ is maximized
		      when $m = \frac{n}{e}$ , and explain why this means $Pr(E)
			      \geq \frac{1}{e}$ for this choice of $m$.
		      \hspace*{\fill}[3 marks]
	\end{enumerate}
\end{que}

\begin{tcolorbox}[breakable]
	\begin{sol}
		Probability that $i^\text{th}$ person is best is $\frac{1}{n}$,
		let him be $X$. All other remaining people are not better than
		$X$. Now, if $i\leq m$, $Pr(E_i) = 0$. For $i\geq m+1$, $X$ is
		selected if and only if the best person from first to
		$m^\text{th}$ person (let him be $M$) is better than everyone
		from $(m+1)^\text{th}$ to $(i-1)^\text{th}$ person. We'll try
		to count the number of permutations of people such that this
		condition is satisfied.

		When all of the $n$ people are arranged in ascending order, let
		$j^\text{th}$ person be $b_j$, the last person (i.e $b_n$) is
		$X$. Suppose the best of first $i-1$ is $b_j$, then $b_j$
		should be in first $m$ people. Remaining everyone in $i-1$
		people is before $b_j$. The number of combinations of first
		$i-1$ people (except $b_j$) is
		\begin{align}
			\C{j-1}{i-1}
		\end{align}
		Total number of permutations when $i^\text{th}$ person is best
		and the best of first $m$ people is $b_j$ is
		\begin{align}
			\C{j-1}{i-2}\cdot m\cdot (i-2)!\cdot(n-i)!
		\end{align}
		Since, $j$ can be in [$i-1$, $n-1$] So the total number of
		permutations is
		\begin{align}
			P & ={\sum_{j=i-1}^{n-1}} {\C{j-1}{i-2}} \cdot m \cdot (i-2)! \cdot (n-i)! \\
			  & = \C{n-1}{i-1}\cdot m \cdot (i-2)! \cdot (n-i)!
		\end{align}
		Total number of permutations is $(n-1)!$, probability that
		$i^\text{th}$ person is selected given he is the best is
		\begin{align}
			Pr(\text{selected } i | i \text{ is best}) & = \frac{\C{n-1}{i-1}\cdot m \cdot (i-2)! \cdot (n-i)!}{(n-1)!} \\
			                                           & = \frac{m}{i-1}
		\end{align}
		Now, $Pr(i\text{ is best}) = \frac{1}{n}$. Therefore,
		\begin{align}
			Pr(\text{selected } i \cap i \text{ is best}) & =
			Pr(\text{selected } i | i \text{ is best})\cdot Pr(i
			\text{ is best})                                  \\
			                                              & =
			\frac{m}{i-1}\cdot \frac{1}{n}
		\end{align}
		Since each of $E_i$ is disjoint,
		\begin{align}
			Pr(E) & = \sum_{j=0}^{n} Pr(E_j)                    \\
			      & = \frac{m}{n}\sum_{j=m+1}^{n} \frac{1}{j-1}
		\end{align}

		\begin{center}
			\begin{tikzpicture}
				\begin{axis}[
						domain=4:15,
						samples=200,
						xmin=4, xmax=15,
						ymin=0, ymax=0.4,
						axis x line=middle, axis y line=middle,
						xlabel={$x$}, ylabel={$y$},
						xtick={4,6,8,10,12,14,15},
						xticklabels={},
						ytick=\empty,
						enlargelimits=true,
						grid=major,
						legend pos=north east,
						width=12cm,
						height=8cm,
						no markers
					]

					% Plotting 1/x
					\addplot[name path=A, domain=4:15, thick, blue] {1/x};
					\addlegendentry{$y = \frac{1}{x}$}

					% Plotting 1/(x-1)
					\addplot[name path=B, domain=4:15, thick, red] {1/(x-1)};
					\addlegendentry{$y = \frac{1}{x-1}$}

					% Define the path at y=0 for filling purposes
					\path[name path=axis] (axis cs:4,0) -- (axis cs:15,0);

					% Shading the area under 1/x
					\addplot [blue, opacity=0.3] fill between [
							of=A and axis,
							soft clip={domain=4:15},
						];

					% Shading the area under 1/(x-1)
					\addplot [red, opacity=0.3] fill between [
							of=B and axis,
							soft clip={domain=4:15},
						];

					% Drawing rectangles under 1/x
					\foreach \m in {4,5,...,14} {
							\addplot+ [ybar interval, samples at={\m, \m+1}, fill=blue, fill opacity=0.3, draw=blue] coordinates {(\m, 0) (\m, {1/\m}) (\m+1, {1/\m}) (\m+1, 0)};
						}

					% Custom labels at specific ticks
					\node at (axis cs:4, -0.05) [anchor=north] {$m$};
					\node at (axis cs:5, -0.05) [anchor=north] {$m+1$};
					\node at (axis cs:9, -0.05) [anchor=north] {$\dots$};
					\node at (axis cs:14, -0.05) [anchor=north] {$n-1$};
					\node at (axis cs:15, -0.05) [anchor=north] {$n$};

				\end{axis}
			\end{tikzpicture}
		\end{center}

		Area under the rectangles is just $\sum_{j=m}^{n-1}\frac{1}{j}$, which is bound by the area of the two curves $\frac{1}{x}$ and $\frac{1}{x-1}$
		\begin{align}
			\int_{m}^{n} \frac{1}{x} dx \leq \sum_{j=m}^{n-1}\frac{1}{j} \leq \int_{m}^{n} \frac{1}{x-1} dx
		\end{align}
		\begin{align}
			\implies \frac{m}{n}(\ln{n} - \ln{m}) \leq Pr(E) \leq \frac{m}{n}(\ln{(n-1)} - \ln{(m-1)})
			\label{eq:conc4}
		\end{align}
		Let $n=mk$, then
		\begin{align}
			\frac{m}{n}(\ln{n} - \ln{m}) & = \frac{1}{k}(\ln{mk} - \ln{m}) \\
			                             & = \frac{\ln{k}}{k} = f(k)
		\end{align}
		This is maximum when $f'(k) = 0$ and $f"(k) < 0$. $f'(k) = \frac{1-\ln{k}}{k^2} = 0$, then $k=e$. Therefore $\frac{m}{n} = e$ i.e $n=\frac{m}{e}$.
		By equation \ref{eq:conc4}, $Pr(E) \geq \frac{1}{e}$ for this choice of $m$ and $n$.

		Hence Proved.
	\end{sol}
\end{tcolorbox}


