% Start writing your answer from here, if you want to use new packages/change something do it in main.tex
\begin{que}
	Suppose that you have computed the mean, median and standard deviation of a
	set of $n$ numbers stored in array $A$ where $n$ is very large. Now, you decide
	to add another number to $A$. Write a python function to update the
	previously computed mean, another python function to update the previously
	computed median, and yet another python function to update the previously
	computed standard deviation. Note that you are not allowed to simply
	recompute the mean, median or standard deviation by looping through all the
	data. You may need to derive formulae for this. Include the formulae and
	their derivation in your report. Note that your python functions should be
	of the following form:

	\begin{verbatim}
	function newMean = UpdateMean(OldMean, NewDataValue, n, A),
	function newMedian = UpdateMedian(OldMedian, NewDataValue, n, A),
	function newStd = UpdateStd(OldMean, OldStd, NewMean, NewDataValue, n, A).
	\end{verbatim}

	Also explain, how would you update the histogram of $A$, if you received
	a new value to be added to $A$? (Only explain, no need to write code.)
	Please specify clearly if you are making any assumptions.
	\hspace*{\fill}[10 marks]
\end{que}

\begin{tcolorbox}
	\begin{sol}
		% Add your solution here
	\end{sol}
\end{tcolorbox}
