\ s s hapter{5}
\begin{task}{Task A}
	Given, PDF of GMM variable $X$ is $f_X = \sum_{i=1}^K p_iP[X_i=x]$. Let
	it's CDF be $F_X$. Then $F_X(x)$ is given by

	\begin{align}
		F_X(x) & = P[X\leq x]                                    \\
		       & = \int_{-\infty}^{x} f_X(t)dt                   \\
		       & = \int_{-\infty}^{x} \sum_{i=1}^K p_iP[X_i=t]dt \\
		       & = \sum_{i=1}^K p_i\int_{-\infty}^{x} P[X_i=t]dt \\
		       & = \sum_{i=1}^K p_iP[X_i\leq x]                  \\
		       & = \sum_{i=1}^K p_iF_{X_i}(x)
	\end{align}
	Where $F_{X_i}(x) = P[X_i\leq x]$ is CDF of $X_i$.

	Now, let CDF of output of the given algorithm be
	$F_\A(x) = P[\A\leq x]$. Since the events that we choose $\A$ to be
	from the distribution $i$ (say $E_i$) are disjoint for $i=1,\dots,k$.
	\begin{align}
		F_\A(x) & = P[\A\leq x]                                \\
		        & = \sum_{i=1}^K P[E_i]\cdot P[\A\leq x | E_i] \\
		        & = \sum_{i=1}^K p_iF_{X_i}(x)                 \\
		        & = F_X(x)
	\end{align}

	We know that PDF of a random variable $X$ with CDF $F_X(x)$ is
	$\frac{\partial F_X}{\partial x}$. Thus,
	\begin{align}
		f_\A(x) & = \frac{\partial F_\A}{\partial x} \\
		        & = \frac{\partial F_X}{\partial x}  \\
		        & = f_X
	\end{align}

	Since $x$ was arbitrary, for every $u\in\mathbb{R}$, $f_\A(u) =
		f_X(u)$. i.e the algorithm indeed samples from the GMM variable's
	distribution.
\end{task}



\begin{task}{Task B}
	Since
	\begin{align}
		\E[X] & = \int_{-\infty}^{\infty} t\cdot P[X=t]dt                    \\
		      & = \int_{-\infty}^{\infty} t\cdot \sum_{i=1}^K p_iP[X_i=t] dt \\
		      & = \sum_{i=1}^K p_i \int_{-\infty}^{\infty} P[X_i = t] dt     \\
		      & = \sum_{i=1}^K p_i \E[X_i]                                   \\
		      & = \sum_{i=1}^K p_i\mu_i
	\end{align}

	Let $\mu = \E[X]$.
	\begin{align}
		\Var[X] & = \int_{-\infty}^{\infty} (t-\mu)^2 P[X=t] dt                    \\
		        & = \int_{-\infty}^{\infty} (t-\mu)^2 \sum_{i=1}^K p_iP[X_i=t] dt  \\
		        & = \sum_{i=1}^K p_i \int_{-\infty}^{\infty} (t-\mu)^2 P[X_i=t] dt \\
		        & = \sum_{i=1}^K p_i \Var[X_i]                                     \\
		        & = \sum_{i=1}^K p_i\sigma_i^2
	\end{align}

	Let $\sigma^2 = \Var[X]$.
	\begin{align}
		\MGF_X(t) & = \int_{-\infty}^{\infty} e^{tX}P[X=x]dx                     \\
		          & = \int_{-\infty}^{\infty} e^{tX} \sum_{i=1}^K p_iP[X_i=x] dx \\
		          & = \sum_{i=1}^K p_i \int_{-\infty}^{\infty} e^{tX}P[X_i=x] dx \\
		          & = \sum_{i=1}^K p_i\MGF_{X_i}(t)                              \\
		          & = \sum_{i=1}^K p_i e^{t\mu_i + \frac{1}{2}t^2\sigma_i^2}
	\end{align}

\end{task}


\begin{task}{Task C}
	\begin{enumerate}
		\item Given $Z=\sum_{i=1}^Kp_iX_i$, where $X_i\sim\N(\mu_i,\sigma_i^2)$

		      \begin{align}
			      \E[Z] & = \E\left[\sum_{i=1}^Kp_iX_i\right] \\
			            & = \sum_{i=1}^Kp_i\E[X_i]            \\
			            & = \sum_{i=1}^Kp_i\mu_i
		      \end{align}

		\item       For $\Var[Z]$,
		      \begin{align}
			      \Var[Z] & = \E[Z^2] - \E[Z]^2                                                                                                                                \\
			              & = \E\left[ \left(\sum_{i=1}^k p_i X_i\right)^2\right] - \left(\sum_{i=1}^k p_i\mu_i\right)^2                                                       \\
			              & = \E\left[\sum_{i=1}^N p_i^2X_i^2\right] + \E\left[2\sum_{i\neq j}p_ip_jX_iX_j\right] - \sum_{i=1}^Kp_i^2\mu_i^2 - 2\sum_{i\neq j}p_ip_j\mu_i\mu_j \\
			              & = \sum_{i=1}^K p_i^2\E[X_i^2] + 2\sum_{i\neq j} p_ip_j \E[X_iX_j] - \sum_{i=1}^Kp_i^2\mu_i^2 - 2\sum_{i\neq j}p_ip_j\mu_i\mu_j                     \\
			      \intertext{Since each $X_i$, $X_j$ ($i\neq j$) are independent}
			              & = \sum_{i=1}^K p_i^2(\sigma_i^2 + \mu_i^2) + 2\sum_{i\neq j} p_ip_j \mu_i\mu_j - \sum_{i=1}^Kp_i^2\mu_i^2 - 2\sum_{i\neq j}p_ip_j\mu_i\mu_j        \\
			              & = \sum_{i=1}^K p_i^2\sigma_i^2
		      \end{align}
		      Thus, $\Var[Z] = \sum_{i=1}^K p_i^2\sigma_i^2$.


		\item Let $\Phi(\mu, \sigma^2) = \frac{1}{\sigma\sqrt{2\pi}}\exp\left(-\frac{(u-\mu)^2}{2\sigma^2}\right) $.

		      For PDF $f_Z(u)$ of $Z$, Let $\hat{\mu} = \sum_{i=1}^K p_i\mu_i$ and
		      $\hat{\sigma}^2 = \sum_{i=1}^K p_i^2\sigma^2$. Then,
		      \begin{align}
			      f_Z(u) & = \Phi(\hat{\mu}, \hat{\sigma}^2)
		      \end{align}
		      This seems intuitive from the above results. Let's try to prove it using induction on $i$. For $i=1$, $Z = p_1X_1$, $\hat{\mu} = p_1\mu_1$, $\hat{\sigma}^2 = p_1^2\sigma_1^2$
		      \begin{align}
			      f_Z(u) & = \Phi (p_1\mu_1, p_1\mu_1^2)      \\
			             & = \Phi (\hat{\mu}, \hat{\sigma}^2)
		      \end{align}

		      Now, let $\forall j \text{ s.t }j<K$ if $Z=\sum_{i=1}^j p_iX_i$ then $f_Z(u) = \Phi(\hat{\mu}, \hat{\sigma}^2) = \Phi(\sum p_i\mu_i, \sum p_i^2\sigma_i^2)$.

		      For $Z = \sum_{i=1}^K p_iX_i$, let $Z' = \sum_{i=1}^{K-1} p_iX_i$, $\hat{\mu}' = \sum_{i=1}^{K-1}p_i\mu_i$ and $\hat{\sigma}'^2 = \sum_{i=1}^{K-1}p_i^2\sigma_i^2$
		      \begin{align}
			      f_Z(u) & = \frac{\partial}{\partial u}P[Z\leq u]                                                                                                        \\
			             & = \frac{\partial}{\partial u}\int_{-\infty}^{\infty} P[x \leq Z' \leq x+dx \text{ and } p_KX_k \leq u-x] dx                                    \\
			             & = \frac{\partial}{\partial u}\int_{-\infty}^\infty f_{Z'}(x)P[p_KX_K\leq u-x]dx                                                                \\
			             & = \int_{-\infty}^\infty f_{Z'}(x)f_{X_K}(u-x)dx                                                                                                \\
			             & = \frac{1}{2\pi\sigma'\sigma_K}\int_{-\infty}^\infty \exp\left(-\frac{(x-\hat{\mu}')^2}{2\sigma'^2} - \frac{(x-\mu_K)^2}{2\sigma_K^2}\right)dx
		      \end{align}
		      This can be simplified as it's an exponential of a quadratic equation,
		      gaussial integral has to be used. This simplifies to the gaussian
		      $\Phi(\hat{\mu}, \hat{\sigma}^2)$. Thus $\forall i\in \mathbb{N}$,
		      $f_Z(u) = \Phi(\sum_{i=1}^Kp_i\mu_i, \sum_{i=1}^Kp_i^2\sigma_i^2)$

		      We've proved that
		      \begin{align}
			      Z & \sim \N\left(\sum_{i=1}^Kp_i\mu_i, \sum_{i=1}^Kp_i^2\sigma_i^2\right) \\
			        & \sim \N(\hat{\mu}, \hat{\sigma}^2)
		      \end{align}

		\item Now, MGF of $Z$, $M_Z(t)$ is
		      \begin{align}
			      M_Z(t) & = \exp\left(\hat{\mu}t + \frac{\hat{\sigma}^2t^2}{2}\right)                             \\
			             & = \exp\left(\sum_{i=1}^K p_i\mu_i t + \frac{\sum_{i=1}^K p_i^2\sigma_i^2 t^2}{2}\right)
		      \end{align}

		\item $X$ and $Z$ are different. $X$ is a mixture of gaussian distributions
		      whereas $Z$ itself is a gaussian distribution. Their PDFs would be
		      different, $X$ would have $i$ peaks whereas $Z$ would have only one
		      peak. $X$ is like a weighted sum of gaussian distributions. Z is a
		      gaussian distribution whose mean and variance are weighted sum of means
		      and variances of other gaussian distributions.

		\item $Z$ follows a gaussian distribution with mean $\sum_{i=1}^K p_i\mu_i$ and variance $\sum_{i=1}^Kp_i^2\sigma_i^2$.
	\end{enumerate}
\end{task}

\begin{task}{Task D(B)}
	Suppose we have two arbitrary random variables $X_1$ and $X_2$. Let's
	try to prove the theorem in one direction first. Suppose that we know
	PDF/PMF of $X_1$ and $X_2$ i.e $f_{X_1}(x)$ and $f_{X_2}(x)$ are equal.
	Let MGF of $X_1$ and $X_2$ be $M_{X_1}(t)$ and $M_{X_2}(t)$
	respectively. Let their common sample space be $S$ (PDFs are equal).
	We have two cases:
	\begin{enumerate}
		\item \textbf{$X_1$ and $X_2$ are finite or discrete}:
		      \begin{align}
			      f_{X_1}(x)            & = f_{X_2}(x) \text{ for all } x \in S            \\
			      \sum e^{tx}f_{X_1}(x) & = \sum e^{tx}f_{X_2}(x) \text{ for all } t \in S \\
			      M_{X_1}(t)            & = M_{X_2}(t) \text{ for all } t \in S
		      \end{align}
		\item \textbf{$X_1$ and $X_2$ are continuous}:
		      \begin{align}
			      f_{X_1}(x)                & = f_{X_2}(x) \text{ for all } x              \\
			      \int_{S} e^{tx}f_{X_1}(x) & = \int_S e^{tx}f_{X_2}(x) \text{ for all } t \\
			      M_{X_1}(t)                & = M_{X_2}(t) \text{ for all } t
		      \end{align}
	\end{enumerate}

	Let's try to prove the other direction, i.e suppose MGFs of $X_1$ and
	$X_2$ are equal, i.e $M_{X_1}(t) = M_{X_2}(t)$. We again have cases
	\begin{enumerate}
		\item \textbf{$X_1$ and $X_2$ are finite or discrete}:
		      \begin{align}
			      M_{X_1}(t)                                     & = M_{X_2}(t) \text{ for all } t \in S               \\
			      \sum_{x\in S} e^{tx}f_{X_1}(x)                 & = \sum_{x\in S} e^{tx}f_{X_2}(x) \text{ for all } t \\
			      \sum_{x\in S} e^{tx} (f_{X_1}(x) - f_{X_2}(x)) & = 0                                                 \\
			      \sum_{x\in S} a_x(e^{t})^x                     & = 0
		      \end{align}
		      where $a_x=f_{X_1}(x)-f_{X_2}(x)$. This has infinitely many solutions, which is only possible if all the coefficients $a_x$ are 0, Thus
		      \begin{align}
			      a_k        & = 0 \text{ for all } k          \\
			      f_{X_1}(x) & = f_{X_2}(x) \text{ for all } x
		      \end{align}
		      Hence, PMFs of $X_1$ and $X_2$ are equal.

		\item \textbf{$X_1$ and $X_2$ are continuous}:
		      \begin{align}
			      M_{X_1}(t)                                       & = M_{X_2}(t) \text{ for all } t \in \mathbb{R} \in S                                   \\
			      \int_{x\in S} e^{tx}f_{X_1}(x)dx                 & = \int_{x\in S} e^{tx}f_{X_2}(x)dx \text{ for all } t \in \mathbb{R}                   \\
			      \int_{x\in S} e^{tx} (f_{X_1}(x) - f_{X_2}(x))dx & = 0                                                  \text{ for all } t \in \mathbb{R} \\
			      \int_{x\in S} a(x)(e^{t})^x                   dx & = 0\text{ for all } t\in \mathbb{R}
		      \end{align}
		      where $a(x)=f_{X_1}(x)-f_{X_2}(x)$. This integral is 0 for all $t\in \mathbb{R}$ if and only if $a(x) = 0$ for all $x\in S$. Thus, $f_{X_1}(x) = f_{X_2}(x)$ for all $x\in S$.
		      Hence, PDFs of $X_1$ and $X_2$ are equal.
	\end{enumerate}
	Hence, we've shown that PDFs/PMFs of distributions (continuous/discrete respectively) are equal iff their MGFs are equal.
\end{task}
