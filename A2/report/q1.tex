\begin{solution}
	% Fill you solution here
	\tcbsubtitle{Task A}
	When $X\sim\Ber(p)$, PGF of $X$ is
	\begin{align}
		G_\Ber(z) & = \E(z^X)                     \\
		          & = \sum_{n=0}^\infty P[X=n]z^n
	\end{align}
	Since $P[X=0]=(1-p)$, $P[X=1] = p$, $P[X=n]=0$ when $n>1$,
	\begin{align}
		G_\Ber(z) & = P[X=0]z^0 + P[X=1]z^1 \\
		          & = (1-p) + pz
		\label{en:pgfber}
	\end{align}

	\tcbsubtitle{Task B}
	When $X\sim\Bin(n,p)$, PMF of $X$ is
	\begin{equation}
		P[X=k] = \binom{n}{k}p^k(1-p)^{n-k} \text{  for  } k\leq n.
	\end{equation}
	and $P[X=k]=0$ for $k>n$.
	\begin{align}
		G_\Bin(z) & = \sum_{k=0}^\infty P[X=k]z^k                \\
		          & = \sum_{k=0}^n \binom{n}{k}p^k(1-p)^{n-k}z^k \\
		          & = \sum_{k=0}^n \binom{n}{k}(pz)^k(1-p)^{n-k} \\
		          & = (1-p + pz)^n.
		\label{en:pgfbin}
	\end{align}
	By equation \ref{en:pgfber}, $G_\Bin(z) = (1-p + pz)^n =
		(G_\Ber(z))^n$. Hence proved.

	\tcbsubtitle{Task D}
	When $X\sim\Geo(p)$, PMF of $X$,
	\begin{equation}
		P[X=k] = (1-p)^{k-1}p
	\end{equation}
	for $k>0$. $P[X=0]=0$. Now, PGF of $X$,
	\begin{align}
		G_\Geo(z) & = \sum_{k=0}^\infty P[X=k]z^k       \\
		          & = \sum_{k=1}^\infty P[X=k]z^k       \\
		          & = \sum_{k=1}^\infty p(1-p)^{k-1}z^k \\
		          & = \sum_{k=1}^\infty pz(z-zp)^{k-1}  \\
		          & = pz\sum_{k=0}^\infty (z-zp)^k      \\
		          & = \frac{pz}{1-z+pz}
	\end{align}

	\tcbsubtitle{Task E}
	By equation \ref{en:pgfbin}, $G_\Bin(z) = (1-p + pz)^n = G_X^{(n,p)}(z)$. For
	$Y\sim\NegBin(n,p)$
	\begin{align}
		P[Y=k] = \binom{k-1}{n-1}p^n(1-p)^{k-n}
		\text{ for } k\geq n
	\end{align}
	Otherwise, $P[Y=k]=0$. PGF of $Y$ is
	\begin{align}
		G_Y^{(n,p)}(z) & = \sum_{k=0}^\infty P[Y=k]z^k                           \\
		               & = \sum_{k=n}^\infty \binom{k-1}{n-1}p^n(1-p)^{k-n}z^k   \\
		               & = \sum_{k=0}^\infty \binom{k+n-1}{n-1}p^n(1-p)^kz^{n+k} \\
		               & = (pz)^n\sum_{k=0}^\infty \binom{k+n-1}{n-1}(z-pz)^k
	\end{align}
	We know $\sum_{k=0}^\infty \binom{k+n-1}{n-1}x^k = (1-x)^{-n}$. Thus
	\begin{align}
		G_Y^{(n,p)}(z) & = (pz)^n(1-z+pz)^{-n}                           \\
		               & = \left((1-p^{-1} + p^{-1}z^{-1})^n\right)^{-1} \\
		               & = (G_X^{(n,p^{-1})}(z^{-1}))^{-1}.
	\end{align}
	Hence Proved.

	\tcbsubtitle{Task G}

	\textbf{To prove:} \textit{Given PGF of a random variable $X$ is $G(z)$, expectation of $X$ i.e $\E(x) = G'(1)$}

	\textbf{Proof:} \begin{align}
		G(z)  & = \E(z^X) = \sum_{k=0}^\infty P[X=k]z^k \\
		G'(z) & = \sum_{k=0}^\infty kP[X=k]z^{k-1}      \\
		G'(1) & = \sum_{k=0}^\infty kP[X=k]             \\
		      & = \E[X]
	\end{align}
	Hence Proved.
	Now, Let's derive means of Bernoulli, Binomial, Geometric and Negative Binomial
	distributions using this:
	\begin{enumerate}
		\item \textbf{Bernoulli Distribution:} Let $X\sim\Ber(p)$,
		      \begin{align}
			      G_\Ber(z)  & = (1-p) + pz \\
			      G'_\Ber(z) & = p          \\
			      G'_\Ber(1) & = p = \E[X]
		      \end{align}
		      Thus, $\E[X] = p$.
		\item \textbf{Binomial Distribution:} Let $X\sim\Bin(n,p)$,
		      \begin{align}
			      G_\Bin(z)  & = (1-p + pz)^n       \\
			      G'_\Bin(z) & = np(1-p + pz)^{n-1} \\
			      G'_\Bin(1) & = np = \E[X]
		      \end{align}
		      Thus, $\E[X] = np$.
		\item \textbf{Geometric Distribution:} Let $X\sim\Geo(p)$,
		      \begin{align}
			      G_\Geo(z)  & = \frac{pz}{1-z+pz}                      \\
			      G'_\Geo(z) & = \frac{p(1-z+pz) - pz(p-1)}{(1-z+pz)^2} \\
			                 & = \frac{p}{(1-z+pz)^2}                   \\
			      G'_\Geo(1) & = \frac{p}{p^2} = \frac{1}{p} = \E[X]
		      \end{align}
		      Thus, $\E[X] = \frac{1}{p}$.
		\item \textbf{Negative Binomial Distribution:} Let $X\sim\NegBin(n,p)$,
		      \begin{align}
			      G_\NegBin(z)  & = \left( \frac{pz}{1-z+pz} \right)^n \\
			      G'_\NegBin(z) & = n\left( \frac{pz}{1-z+pz}
			      \right)^{n-1}\left(\frac{p}{(1-z+pz)^2}\right)       \\
			      G'_\NegBin(1) & = \frac{n}{p} = \E[X]
		      \end{align}
		      Thus, $\E[X] = \frac{n}{p}$.
	\end{enumerate}
\end{solution}
