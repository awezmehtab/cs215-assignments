\chapter{1}

\begin{task}{ Task A }

	When $X\sim\Ber(p)$, PGF of $X$ is

	\begin{align}
		G_\Ber(z) & = \E(z^X)                     \\
		          & = \sum_{n=0}^\infty P[X=n]z^n
	\end{align}
	Since $P[X=0]=(1-p)$, $P[X=1] = p$, $P[X=n]=0$ when $n>1$,
	\begin{align}
		G_\Ber(z) & = P[X=0]z^0 + P[X=1]z^1 \\
		          & = (1-p) + pz
		\label{en:pgfber}
	\end{align}

\end{task}


\begin{task}{ Task B }
	When $X\sim\Bin(n,p)$, PMF of $X$ is
	\begin{equation}
		P[X=k] = \binom{n}{k}p^k(1-p)^{n-k} \text{  for  } k\leq n.
	\end{equation}
	and $P[X=k]=0$ for $k>n$.
	\begin{align}
		G_\Bin(z) & = \sum_{k=0}^\infty P[X=k]z^k                \\
		          & = \sum_{k=0}^n \binom{n}{k}p^k(1-p)^{n-k}z^k \\
		          & = \sum_{k=0}^n \binom{n}{k}(pz)^k(1-p)^{n-k} \\
		          & = (1 - p + pz)^n
		\label{en:pgfbin}
	\end{align}
	By equation \ref{en:pgfber},
	\begin{align}
		G_{\text{Bin}}(z) = (1-p + pz)^n = (G_\Ber(z))^n
	\end{align}
	% Hence proved.
\end{task}
\begin{task}{Task C}
	By the definition of PGF,
	\begin{align}
		G(z) = \sum_{n=0}^\infty P[X_1=n]z^n
	\end{align}

	Let $\left(G(z)\right)^k = \sum_{n=0}^\infty a_nz^n$. Now, $G_\Sigma(z)$ is
	\begin{align}
		G_\Sigma(z) & = \sum_{n=0}^\infty P[X=n]z^n                        \\
		            & = \sum_{n=0}^\infty P[X_1+\dots+X_k=n]z^n            \\
		            & = \sum_{n=0}^\infty \sum P[X_1=i_1,\dots,X_k=i_k]z^n
		\intertext{where $i_1+\dots+i_k=n$.}
		            & = \sum_{n=0}^\infty \sum P(i_1)\dots P(i_n)z^{n}
		\label{en:pgfsigma}
	\end{align}

	Now, $G(z) = \sum_{n=0}^\infty P(n)z^n$.
	And since $a_n$ is coefficient of $z^n$ in $\left(G(z)\right)^k = (\sum_{n=0}^\infty P(n)z^n)^k$.
	\begin{align}
		a_n & = \sum P(i_1)P(i_2)\dots P(i_k) \text{ where } i_1+\dots+i_k=n
	\end{align}
	By equation \ref{en:pgfsigma}
	\begin{align}
		G_\Sigma(z) & = \sum_{n=0}^\infty a_nz^n = \left(G(z)\right)^k
		\label{en:pgfsigmak}
	\end{align}

	Hence Proved.
\end{task}

\begin{task}{ Task D }
	When $X\sim\Geo(p)$, PMF of $X$,
	\begin{equation}
		P[X=k] = (1-p)^{k-1}p
	\end{equation}
	for $k>0$. $P[X=0]=0$. Now, PGF of $X$,
	\begin{align}
		G_\Geo(z) & = \sum_{k=0}^\infty P[X=k]z^k       \\
		          & = \sum_{k=1}^\infty P[X=k]z^k       \\
		          & = \sum_{k=1}^\infty p(1-p)^{k-1}z^k \\
		          & = \sum_{k=1}^\infty pz(z-zp)^{k-1}  \\
		          & = pz\sum_{k=0}^\infty (z-zp)^k      \\
		          & = \frac{pz}{1-z+pz}
		\label{en:pgfgeo}
	\end{align}
\end{task}



\begin{task}{ Task E }
	By equation \ref{en:pgfbin},
	\begin{align}
		G_\Bin(z) = (1-p + pz)^n = G_X^{(n,p)}(z).
	\end{align}

	For $Y\sim\NegBin(n,p)$, $Y$ represents the number of independent coin
	throws required to get $n$ heads of a coin. Let $X_i$ represents the
	number of throws of coin required after getting $(i-1)^\text{th}$ head
	to get the $i^\text{th}$ head. Since all of the coin throws are
	independent, the outcome of a given throw doesn't depend on the
	previous coins' output. Thus, $X_i$ is just the number of throws to get
	a head when a coin in thrown, where each $X_i\sim\Geo(p)$ since each
	coin is same with probability of getting head as $p$.

	$Y$ can be written as $Y=X_1+X_2+\dots+X_k$. Using equations
	\ref{en:pgfsigmak} and \ref{en:pgfgeo},
	\begin{align}
		G_Y^{(n,p)}(z) & = (G_\Geo(z))^n                    \\
		               & = \left(\frac{pz}{1-z+pz}\right)^n \\
		\label{en:pgfnegbin}
	\end{align}
	\begin{align}
		G_X^{(n,p^{-1})}(z^{-1})                   & =
		\left(1-\frac{1}{p} + \frac{1}{pz}\right)^n                   \\
		                                           & =
		\left(\frac{1-z+pz}{pz}\right)^n
		\\
		\left(G_X^{(n,p^{-1})}(z^{-1})\right)^{-1} & =
		\left(\frac{pz}{1-z+pz}\right)^n
		\\
		                                           & = G_Y^{(n,p)}(z)
	\end{align}

	Hence Proved.
\end{task}



\begin{task}{ Task F }
	For $Y\sim\NegBin(n,p)$,
	\begin{align}
		P[Y=k] = \binom{k-1}{n-1}p^n(1-p)^{k-n}
		\text{ for } k\geq n
	\end{align}
	Otherwise, $P[Y=k]=0$. PGF of $Y$ is
	\begin{align}
		G_Y^{(n,p)}(z) & = \sum_{k=0}^\infty P[Y=k]z^k                           \\
		               & = \sum_{k=n}^\infty \binom{k-1}{n-1}p^n(1-p)^{k-n}z^k   \\
		               & = \sum_{k=0}^\infty \binom{k+n-1}{n-1}p^n(1-p)^kz^{n+k} \\
		               & = (pz)^n\sum_{k=0}^\infty \binom{k+n-1}{n-1}(z-pz)^k
	\end{align}

	Using equation \ref{en:pgfnegbin},
	\begin{align}
		\left(\frac{pz}{1-z+pz}\right)^n & = (pz)^n\sum_{k=0}^\infty
		\binom{k+n-1}{n-1}(z-pz)^k                                   \\
		\left(1+pz-z\right)^{-n}         & = \sum_{k=0}^\infty
		\binom{k+n-1}{n-1}(z-pz)^k
	\end{align}
	Since $z$, $p$ are arbitrary, let $pz-z = x$.
	\begin{align}
		(1+x)^{-n} & = \sum_{r=0}^\infty (-1)^r\binom{r+n-1}{n-1}x^r =
		\sum_{r=0}^\infty (-1)^r\binom{n+r-1}{r}x^r
	\end{align}
	Now,
	\begin{align}
		(-1)^r\binom{n+r-1}{r} & = (-1)^r\frac{(n+r-1)(n+r-2)\cdots n}{r!} \\
		                       & = \frac{(-n)(-n-1)\cdots (-n-r+1)}{r!}    \\
		                       & = \binom{-n}{r}
	\end{align}

	Thus,
	\begin{align}
		(1+x)^{-n} = \sum_{r=0}^\infty (-1)^r\binom{n-r+1}{r} x^r=
		\sum_{r=0}^\infty \binom{-n}{r}x^r
	\end{align}

	Hence proved.
\end{task}




\begin{task}{ Task G }

	\textbf{To prove:} \textit{Given PGF of a random variable $X$ is
		$G(z)$, expectation of $X$ i.e $\E(x) = G'(1)$}

	\textbf{Proof:} \begin{align}
		G(z)  & = \E(z^X) = \sum_{k=0}^\infty P[X=k]z^k \\
		G'(z) & = \sum_{k=0}^\infty kP[X=k]z^{k-1}      \\
		G'(1) & = \sum_{k=0}^\infty kP[X=k]             \\
		      & = \E[X]
	\end{align}
	Hence Proved.
	Now, Let's derive means of Bernoulli, Binomial, Geometric and Negative Binomial
	distributions using this:
	\begin{enumerate}
		\item \textbf{Bernoulli Distribution:} Let $X\sim\Ber(p)$,
		      \begin{align}
			      G_\Ber(z)  & = (1-p) + pz \\
			      G'_\Ber(z) & = p          \\
			      G'_\Ber(1) & = p = \E[X]
		      \end{align}
		      Thus, $\E[X] = p$.
		\item \textbf{Binomial Distribution:} Let $X\sim\Bin(n,p)$,
		      \begin{align}
			      G_\Bin(z)  & = (1-p + pz)^n       \\
			      G'_\Bin(z) & = np(1-p + pz)^{n-1} \\
			      G'_\Bin(1) & = np = \E[X]
		      \end{align}
		      Thus, $\E[X] = np$.
		\item \textbf{Geometric Distribution:} Let $X\sim\Geo(p)$,
		      \begin{align}
			      G_\Geo(z)  & = \frac{pz}{1-z+pz}                      \\
			      G'_\Geo(z) & = \frac{p(1-z+pz) - pz(p-1)}{(1-z+pz)^2} \\
			                 & = \frac{p}{(1-z+pz)^2}                   \\
			      G'_\Geo(1) & = \frac{p}{p^2} = \frac{1}{p} = \E[X]
		      \end{align}
		      Thus, $\E[X] = \frac{1}{p}$.
		\item \textbf{Negative Binomial Distribution:} Let $X\sim\NegBin(n,p)$,
		      \begin{align}
			      G_\NegBin(z)  & = \left( \frac{pz}{1-z+pz} \right)^n \\
			      G'_\NegBin(z) & = n\left( \frac{pz}{1-z+pz}
			      \right)^{n-1}\left(\frac{p}{(1-z+pz)^2}\right)       \\
			      G'_\NegBin(1) & = \frac{n}{p} = \E[X]
		      \end{align}
		      Thus, $\E[X] = \frac{n}{p}$.
	\end{enumerate}

\end{task}
