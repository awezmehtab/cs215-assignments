\begin{solution}


	\tcbsubtitle{Task A}

	\textbf{To prove: }

	\emph{Let $X$ be a continuous real-valued random variable with CDF  : $\mathbb{R} \rightarrow [0, 1]$. Assume that
		$F_X$ is invertible. Then the random variable $Y := F_X (X) \in [0, 1]$ is uniformly distributed in $[0, 1]$}

	\textbf{Proof:}\\
	$F_X$ by definition can also be written as
	\begin{align}
		F_X(x) = P(X\leq x)
	\end{align}

	Define a new random variable $Y$,
	\begin{align}
		Y =F_X(X)
	\end{align}

	Y is the result of applying CDF $F_X$ to the random variable $X$. To
	prove the theorem, assume $y\in [0,1]$. So, the probablity that $Y \leq
		y$ is:
	\begin{align}
		P(Y\leq y) = P(F_X(X)\leq y)
	\end{align}

	It is assumed that $F_X(x)$ is invertible, so,
	\begin{align}
		P(Y\leq y) = P(X\leq F_X^{-1}(y))
	\end{align}

	which is basically, probablity that $X$ is less that or equal to $F_X^{-1}(y)$. This can be written in the CDF form, which is $F_X(F_X^{-1}(y))$. So,
	\begin{align}
		P(Y\leq y) = P(X\leq F_X^{-1}(y)) = F_X(F_X^{-1}(y)) = y
	\end{align}

	So,
	\begin{align}
		P(Y\leq y) = y
	\end{align}

	where $y\in [0,1]$, which is the CDF of uniform distributon in $[0,1]$.
	So, Y is a uniform distributon in $[0,1]$ regardless of $X$.



	\tcbsubtitle{Task B}

	According to the theorem proved above, CDF of any random variable $X$
	mapped with itself gives a uniform random variable $Y$ in $[0,1]$. So,
	let $Y\sim \text{Uniform}(0,1)$. Then for any random variable $X$,
	\begin{align}
		F_X(X) & = Y           \\
		X      & = F_X^{-1}(Y)
	\end{align}

	\textbf{Algorithm $\mathcal{A}$:}
	\begin{enumerate}
		\item Input: A sample $y$ from the uniform distributon on $[0,1]$.
		\item Transformation:
		      \begin{itemize}
			      \item Apply the inverse CDF to $y$ to compute a sample $u$.
			      \item Define $\mathcal{A}(u) = u = F_X^{-1}(y)$
		      \end{itemize}
		\item Output: The random variable $U = F_X^{-1}(Y)$
	\end{enumerate}

	This gives us the correct required random variables as, CDF of U is $F_U(u)$,
	\begin{align}
		P(U\leq u ) & = P(F_X(Y) \leq u)            \\
		F_U(u)      & =  P(F_X(F_X^{-1}(X) \leq u)) \\
		F_U(u)      & =  P(X \leq u)                \\
		F_U(u)      & =  F_X(u)                     \\
	\end{align}

	$U$ and $X$ have the same CDF, which was initially required.

	\textbf{Proof:}\\
	$F_X$ by definition can also be written as
	\begin{align}
		F_X(x) = P(X\leq x)
	\end{align}

	Define a new random variable $Y$,
	\begin{align}
		Y =F_X(X)
	\end{align}

	Y is the result of applying CDF $F_X$ to the random variable $X$.
	To prove the theorem, assume $y\in [0,1]$. So, the probability that $Y \leq y$ is:
	\begin{align}
		P(Y\leq y) = P(F_X(X)\leq y)
	\end{align}

	It is assumed that $F_X(x)$ is invertible, so,
	\begin{align}
		P(Y\leq y) = P(X\leq F_X^{-1}(y))
	\end{align}

	which is basically, probability that $X$ is less that or equal to $F_X^{-1}(y)$. This can be written in the CDF form, which is $F_X(F_X^{-1}(y))$. So,
	\begin{align}
		P(Y\leq y) = P(X\leq F_X^{-1}(y)) = F_X(F_X^{-1}(y)) = y
	\end{align}

	So,
	\begin{align}
		P(Y\leq y) = y
	\end{align}

	where $y\in [0,1]$, which is the CDF of uniform distributon in $[0,1]$.
	So, Y is a uniform distributon in $[0,1]$ regardless of $X$.


	\tcbsubtitle{Task B}
	According to the theorem proved above, CDF of any random variable $X$
	mapped with itself gives a uniform random variable $Y$ in $[0,1]$. So,
	let $Y\sim \text{Uniform}(0,1)$. Then for any random variable $X$,
	\begin{align}
		F_X(X) & = Y           \\
		X      & = F_X^{-1}(Y)
	\end{align}

	\textbf{Algorithm $\mathcal{A}$:}
	\begin{enumerate}
		\item Input: A sample $y$ from the uniform distributon on $[0,1]$.
		\item Transformation:
		      \begin{itemize}
			      \item Apply the inverse CDF to $y$ to compute a sample $u$.
			      \item Define $\mathcal{A}(u) = u = F_X^{-1}(y)$
		      \end{itemize}
		\item Output: The random variable $U = F_X^{-1}(Y)$
	\end{enumerate}

	This gives us the correct required random variables as, CDF of U is $F_U(u)$,
	\begin{align}
		P(U\leq u ) & = P(F_X(Y) \leq u)            \\
		F_U(u)      & =  P(F_X(F_X^{-1}(X) \leq u)) \\
		F_U(u)      & =  P(X \leq u)                \\
		F_U(u)      & =  F_X(u)                     \\
	\end{align}
	$U$ and $X$ have the same CDF, which was initially required.



	\tcbsubtitle{Task E(B)}
	We have a random variable $X$ which can take values from
	$\{-h,-h+2,\ldots,h-2,h\}$. Each ball makes $h$ random binary
	decisions(left or right) as it descends. If we let $Y$ be the number of
	times the ball moves right, the final position of the ball will be
	given by,
	\begin{align}
		X = -h+2Y
	\end{align}

	where $Y$ is a \textbf{binomial variable} because in simple terms it is the summation of $h$ bernoulli decisions each with probability $\frac{1}{2}$.
	\begin{align}
		Y \sim \Bin(h,\frac{1}{2})
	\end{align}

	For a particular pocket $X = 2i$, the corresponding value of Y is:
	\begin{align}
		Y = \frac{h+2i}{2}
	\end{align}

	Thus, the probability that the ball lands in the pocket $X=2i$ is the
	probability that $Y = \frac{h+2i}{2}$. Using Binmial distribution, this
	is:
	\begin{align}
		P_h[X=2i]=P_h\left[Y=\frac{h+2i}{2}\right]=\binom{h}{\frac{h+2i}{2}}\left(\frac{1}{2}\right)^h
	\end{align}

	This is the \textbf{closed form expression for $P_h[X=2i]$}

	Now, we need to show $P_h[X=2i]$ approximates to normal distribution for very large $h$.\\
	Using \textbf{stiriling's approximation} for large $n$, which states that:
	\begin{align}
		n!\approx \sqrt[]{2\pi n}\left(\frac{n}{2}\right)^n
	\end{align}
	We can convert the factorials in the binomial coefficient:
	\begin{align}
		\binom{h}{r} \approx \frac{h!}{r!(h-r)!}
	\end{align}
	Using stirlings approximation, we have,
	\begin{align}
		\binom{h}{y} = \frac{\sqrt[]{2\pi h}\left(\frac{h}{e}\right)^h}{\sqrt[]{2\pi y}\left(\frac{y}{e}\right)^y\cdot \sqrt[]{2\pi(h-y)}\left(\frac{h-y}{e}\right)^{h-y}}
	\end{align}
	where $y = \frac{h+2i}{2}$
	For large $h$, we can simplify this assuming small $i$ (relative to $h$). In particular, $\frac{h+2i}{2}$ can be written as $\frac{h}{2}$, leading to:
	\begin{align}
		\binom{h}{\frac{h+2i}{2}} \approx \frac{2^h}{\sqrt[]{\pi h}}e^{-\frac{2i^2}{h}}
	\end{align}
	Substituting it back in $P_h$ gives:
	\begin{align}
		P_h[X=2i]\approx \frac{2^h}{\sqrt[]{\pi h}}e^{-\frac{2i^2}{h}}\left(\frac{1}{2}\right)^h
	\end{align}
	Simplifying the powers of 2 gives:
	\begin{align}
		P_h[X=2i]\approx \frac{1}{\sqrt[]{\pi h}}e^{-\frac{2i^2}{h}}
	\end{align}
	which is basically normal distribution with $\mu = 0$ and $\sigma^2=h/2$
\end{solution}
