\vspace{1.2cm}
\section{Non-parametric regression}
\subsection{1. Report Bandwidth Corresponding to Minimum Estimated Risk}

After running the Nadaraya-Watson kernel regression using the Epanechnikov and Gaussian kernel and performing cross-validation for bandwidth selection, the optimal bandwidth corresponding to the minimum estimated risk is:

\[
\textbf{Optimal Bandwidth of Gaussian kernel: 0.180}
\]
\[
\textbf{Optimal Bandwidth of Gaussian kernel: 0.164}
\]

\subsection{2. Comment on Similarities and Differences Due to Choice of Different Kernel Functions}

\subsubsection{Similarities}
\begin{itemize}
    \item \textbf{General Functionality:} Both kernels assign weights to data points based on their distance from the query point, resulting in similar predictions in regions with high data density.
    \item \textbf{Smoothing:} As the bandwidth increases, all kernel functions produce smoother estimates. At very large bandwidths, all kernels oversmooth the data, giving too much influence to distant points.
    \item \textbf{Cross-validation Behavior:} Both kernels display a similar behavior during cross-validation, and the corresponding risk curves follow the same trend with bandwidth changes.
\end{itemize}

\subsubsection{Differences}
\begin{itemize}
    \item \textbf{Shape of the Weights:}
    \begin{itemize}
        \item \textbf{Epanechnikov Kernel:} This kernel assigns zero weight to points farther than the bandwidth due to its quadratic form, creating a more localized effect.
        \item \textbf{Gaussian Kernel:} This kernel assigns non-zero weight to every point, regardless of distance, due to its exponential decay. It results in smoother estimates, but it is more sensitive to distant points.
    \end{itemize}
    
    \item \textbf{Sensitivity to Outliers:}
    \begin{itemize}
        \item \textbf{Epanechnikov Kernel:} This kernel is more resilient to outliers because they assign zero or reduced weight to distant points, decreasing the influence of outliers on the prediction.
        \item \textbf{Gaussian Kernel:} The Gaussian kernel is more prone to incorporating outliers, as it assigns non-zero weights even to far-away points, making it less resilient in the presence of outliers.
    \end{itemize}
    \item \textbf{Plots}
    \begin{itemize}
        \item \textbf{Epanechnikov Kernel:} This kernel produces more precise and localized estimates, with a good balance between bias and variance when using the optimal bandwidth.
        \item \textbf{Gaussian Kernel:} The Gaussian kernel leads to smoother curves but gives undue influence to distant points, which can result in overfitting or oversmoothing depending on the bandwidth.
    \end{itemize}
\end{itemize}





