\section{Finding optimal bandwidth}

a)We are tasked with proving the following equation for the histogram estimator:

\begin{align}
\int \hat{f}(x)^2 \, dx = \frac{1}{n^2 h} \sum_{j=1}^{m} v_j^2
\end{align}

where \( \hat{f}(x) \) is the histogram estimator, \( v_j \) is the number of points in the \( j \)-th bin, \( n \) is the total number of points, \( h \) is the bin width, and \( m \) is the total number of bins.
\\

The histogram estimator \( \hat{f}(x) \) is given by:

\begin{align}
\hat{f}(x) = \sum_{j=1}^{m} \frac{\hat{p}_j}{h} I_{[x \in B_j]}
\end{align}

where \( \hat{p}_j = \frac{v_j}{n} \) is the estimated probability that a point falls in the \( j \)-th bin, and \( I_{[x \in B_j]} \) is the indicator function, which is 1 if \( x \in B_j \), and 0 otherwise.
\\
Now, square \( \hat{f}(x) \):

\begin{align}
\hat{f}(x)^2 = \left( \sum_{j=1}^{m} \frac{\hat{p}_j}{h} I_{[x \in B_j]} \right)^2
\end{align}

Since the bins \( B_j \) are non-overlapping, the cross terms vanish, and we are left with:

\begin{align}
\hat{f}(x)^2 = \sum_{j=1}^{m} \left( \frac{\hat{p}_j}{h} \right)^2 I_{[x \in B_j]}
\end{align}


Integrating \( \hat{f}(x)^2 \) over the entire domain:

\begin{align}
\int \hat{f}(x)^2 \, dx = \int \sum_{j=1}^{m} \left( \frac{\hat{p}_j}{h} \right)^2 I_{[x \in B_j]} \, dx
\end{align}

Because the bins \( B_j \) are disjoint, the integral breaks down into a sum of integrals over each bin:

\begin{align}
\int \hat{f}(x)^2 \, dx = \sum_{j=1}^{m} \int_{B_j} \left( \frac{\hat{p}_j}{h} \right)^2 \, dx
\end{align}

The length of each bin is  $h$ and $ \hat{p}_{j}/h $ is independent of x, so the integral over each bin $B_j$  is:

\begin{align}
\int_{B_j} 1 \, dx = h
\end{align}

Thus, we get:

\begin{align}
\int \hat{f}(x)^2 \, dx = \sum_{j=1}^{m} \left( \frac{\hat{p}_j}{h} \right)^2 h = \frac{1}{h} \sum_{j=1}^{m} \hat{p}_j^2
\end{align}
\\

Recall that \( \hat{p}_j = \frac{v_j}{n} \), so we can substitute \( \hat{p}_j^2 \) as:

\begin{align}
\hat{p}_j^2 = \left( \frac{v_j}{n} \right)^2 = \frac{v_j^2}{n^2}
\end{align}

Thus, the integral becomes:

\begin{align}
\int \hat{f}(x)^2 \, dx = \frac{1}{h} \sum_{j=1}^{m} \frac{v_j^2}{n^2}
\end{align}

Factor out \( \frac{1}{n^2 h} \):

\begin{align}
\int \hat{f}(x)^2 \, dx = \frac{1}{n^2 h} \sum_{j=1}^{m} v_j^2
\end{align}
\\
Hence, proved.
\\
\\
\textbf{1.2}
\\
a)The estimated probabilities for all bins $\hat{p}_j$ is given as:
\begin{align*}
p_1 & = 0.2059, \:
p_2 = 0.4882 \\
p_3 & = 0.0471, \:
p_4 = 0.0412 \\
p_5 & = 0.1353, \:
p_6  = 0.0588 \\
p_7 & = 0.0059, \:
p_8 = 0.0000 \\
p_9 & = 0.0118, \:
p_{10} = 0.0059
\end{align*}
\\
b) The probability distribution is underfit. \\
c) The optimal binwidth h is 0.06835999999999999
\\
d) The plot seems to give a better analysis of the distribution. It  offers significantly more detail and resolution compared to the 10-bin histogram. Each bin covers a narrower range of data values, leading to a more refined visualization that can better capture small variations in the dataset
